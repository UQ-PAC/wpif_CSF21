\section{Motivation}


% \qquad
% \begin{minipage}{0.45\linewidth}
% \begin{lstlisting}
% initially z = 0 $\wedge$ x = 0
% sync write(dataType secret) :
%   while ($\neg$ $\casInstn$(z, 0, 1)) {
%     while (z $\neq$ 0) {}
%   }
%   x := secret;
%   ...
%   x := 0;
%   z := 0;

% sync read : dataType
%   dataType y;
%   if ($\casInstn$(z, 0, 2)) {
%     y := x;
%     z := 0;
%   return y;
%   }

% write(dataType data) :
%   x := data;

% read : dataType
%   return x;
% \end{lstlisting}
% \end{minipage}
% \hfill
\begin{center}
\begin{minipage}{0.45\linewidth}
\begin{lstlisting}
initially z = 0 $\wedge$ x = 0
sync write(dataType secret) :
  while ($\neg$ $\casInstn$(z, 0, 1)) {
    while (z $\neq$ 0) {}
  }
  x := secret;
  ...
  x := 0; // flush;
  z := 0;

sync read : dataType
  dataType y;
  if ($\casInstn$(z, 0, 2)) {
    y := x;
    z := 0;
  return y;
  }

write(dataType data) :
  x := data;

read : dataType
  return x;
\end{lstlisting}
\end{minipage}
\end{center}



\begin{itemize}
\item We require that the secret is not \verb|x= secret| is not leaked
  from the write to the read (provided 
\item The security policy fails under persistent memory, even if the
  underlying model is SC 
\item If the system crashes after \verb|z:=0| has executed, upon
  recovery, we could have \verb|z = 0| and \verb|x = secret| causing
  the security policy to be broken
  
\item This means that we need to insert a flush
  instruction between the \verb|x := 0| and \verb|z := 0| to ensure a
  {\em persistent invariant} $z = 0 \implies x = 0$ 
\item For simplicity, we study the problem under PSC (Khyzha and
  Lahav)
\end{itemize}


\paragraph{Idea 1} Extend the security policy CSF paper directly with WPs for weak memory
Defining $wp'$ in the right way would be the hardest challenge
\begin{itemize}
\item Option 1: Model persistence buffers etc, directly (ugly)
\item Option 2: Assertions that capture the Khyzha/Lahav semantics (wp may not play so nice) 
\end{itemize}

\paragraph{Idea 2} use existing semantics to encode persistent invariants with security policies
- would only be worthwhile if we were to consider Px86-TSO

\section{WP-based treatment of PSC} 

We have previously explored the possibility of extending wp to support weak memory concepts, specifically targeting an extension to the CSF paper. However, the approach became too intricate when considering the ARMv8 memory model, as a significant amount of bookkeeping was required to check all possible execution orders. Persistent memory (and perhaps TSO) might be simple enough to capture though.

I suspect we can come up with suitable over-approximations of the PSC state, sufficient to phrase the properties of interest without modelling the full buffers. For instance, the predicate language could be extended to refer to variables in the persistent memory, e.g. $x_p$ for variable $x$. These predicates would then be interpreted such that a state satisfies it if only if it does so directly and all states reachable via persist steps (PERSIST-W, PERSIST-FO) satisfy it as well. There would be a similar annotation that could be introduced for flush-opt instructions, e.g. $x_{fo}$, with some interpretation based on the relative placement of an $FO$ and a write in the buffer.

The wp rules would then look something like:

\begin{itemize}
\item $wp(flush~x)~Q := Q[x/x_p]$
\item $wp(read~x~v)~Q := x = v \implies Q$ 
\item $wp(write~x~v)~Q := Q[v/x] \land Q[v/x][v/x_p]$ (maybe?)
\end{itemize}
Its also necessary to determine what rely/guarantee specification looks like in this setting. Their interactions with the persistent memory have to be constrained to get soundness plus some axioms to make all of the annotations make sense. For instance, the environment behaviour shouldn't be conditional on the persistent state.

Overall, seems like it should be relatively easy to encode in the wp infrastructure and could be embedded into the CSF Isabelle theories. However, future extensions to support weak memory are not clear. I have a few ideas for wp for TSO, but anything weaker rapidly becomes a problem.

\section{RIF Based}

The complexity of the wp approach for weak memory motivated the
separate weak memory interference check
\url{https://staff.itee.uq.edu.au/kirsten/publications/fm2021.pdf}.

\nc{This
approach requires a static analysis to identify reorderable pairs of
instructions and then requires a proof of interference freedom for
each. I'm not sure how to combine this with persistent memory at the
moment, but I'd hope it would be possible to preserve the wp approach
above with some arbitrary weak memory model using this technique.}

\subsection{Examples}
 
  (taken from initial submission of CSF'21 paper which included WMM considerations)

\paragraph{where reordering of instructions invalidates the PO}
-  Consider the following wp reasoning using $wpi\!f$ where $\{x,y,z\}\subseteq Global$ and
$\cL(x)=\cL=true$ and $\cL(y)= (z=0)$.


\begin{lstlisting}
{true}
  z := 0;
{z=0}      <-----  PO($\beta$)
  x := y;  <-----  $\beta$ 
{true}
  z := 1;  <----- $\alpha\in later(\beta)$
{true}
\end{lstlisting}


While this evaluates as being secure, since $x:=y$ and $z:=1$ do not
share any variables, they can be reordered under a weak memory
model. The reasoning using $wpi\!f$ in this case would indicate a
possible information leak. \bd{Not sure we should consider a
  reordering based semantics. But this example may be useful for Px86-TSO }

\begin{lstlisting}
{false}
  z := 0;
{false}
  z := 1;  <------- $\alpha\in later(\beta)$  (reordered)
{z=0}      <------- PO($\beta$)
  x := y;  <------- $\beta$
{true}
\end{lstlisting}
The leak occurs when a concurrent thread, seeing $z$ set to 1, writes a high value to $y$ before $y$'s value is read into the low variable $x$.


\paragraph{ where reordering of instructions affects the stability of PO}

The checks of stability under $\cR$ on the proof obligation under substitution ($stable_R(wp_Q(c,Q))$) can depend on the order in which substitution is applied. That is, $stable_R(wp_Q(\alpha,Q))\neq stable_R(wp_Q(\beta,Q))$ even if $wp_Q(\alpha;\beta, Q)=wp_Q(\beta;\alpha, Q)$.

Consider the following weakest precondition reasoning using $wp_Q$ over a program from a point where the outstanding proof obligation is $x \leq c$.
  
\begin{lstlisting}
 {n $\leq$ 0}   <----- $PO(\beta)[x\leftarrow n][c \leftarrow 0]$
  c:=0; 
  {n $\leq$ c}  <----- $wp_Q(x:=n, PO(\beta)) = PO(\beta)[x\leftarrow n]$
  x:=n;
  {x $\leq$ c}  <----- $PO(\beta)$
$\beta$ 
\end{lstlisting}
If the rely condition $\cR$ is $x \leq c \land x'\leq c' \land c'=c \land n'=n$ then obviously all the predicates in the reasoning are stable under $\cR$. However, since $x:=n$ and $c:=0$ do not share any variables, they can be reordered under a weak memory model. The code fragment with its reasoning may take the following form,

\begin{lstlisting}
 {n $\leq$ 0}    <------ $PO(\beta)[c\leftarrow 0][n\leftarrow 0]$
  x:=n; 
  {x $\leq$ 0}   <------ $wp_Q(c:=0, PO(\beta)) = PO(\beta)[c\leftarrow 0]$
  c:= 0;
  {x $\leq$ c}   <------ $PO(\beta)$
  $\beta$
\end{lstlisting}
where the middle predicate $x \leq 0$ is not stable under $\cR$. For example, if $c$ were 10 then the environment could set $x$ to any value up to and including 10, violating $x \leq 0$.  Hence, although the original program is secure under $wpi\!f_{RG}$ the latter reordered program is not.



%%% Local Variables:
%%% mode: latex
%%% TeX-master: "main"
%%% End:
